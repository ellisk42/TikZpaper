\documentclass{article}

% if you need to pass options to natbib, use, e.g.:
% \PassOptionsToPackage{numbers, compress}{natbib}
% before loading nips_2016
%
% to avoid loading the natbib package, add option nonatbib:
% \usepackage[nonatbib]{nips_2016}


\usepackage{dsfont}

% to compile a camera-ready version, add the [final] option, e.g.:
\usepackage{nips_2017}

\usepackage[utf8]{inputenc} % allow utf-8 input
\usepackage[T1]{fontenc}    % use 8-bit T1 fonts
\usepackage{hyperref}       % hyperlinks
\usepackage{url}            % simple URL typesetting
\usepackage{booktabs}       % professional-quality tables
\usepackage{amsfonts}       % blackboard math symbols
\usepackage{nicefrac}       % compact symbols for 1/2, etc.
\usepackage{microtype}      % microtypography

\usepackage{listings}
\usepackage{amsthm}
% use Times
\usepackage{times}
% For figures
\usepackage{graphicx} % more modern
%\usepackage{epsfig} % less modern
\usepackage{subfig} 
\usepackage{fancyvrb}


\usepackage{caption}
\usepackage{subcaption}

\fvset{fontsize=\footnotesize}

\usepackage{amssymb}
\usepackage{listings}
\usepackage{wrapfig}
\usepackage{tabularx}


\usepackage{verbatim}
 \usepackage{booktabs}
 % For algorithms
\usepackage{algorithm}
\usepackage{algorithmic}
\usepackage{tikz}
% As of 2011, we use the hyperref package to produce hyperlinks in the
% resulting PDF.  If this breaks your system, please commend out the
% following usepackage line and replace \usepackage{icml2016} with
% \usepackage[nohyperref]{icml2016} above.
\usepackage{amsmath}
\usepackage{hyperref}
\DeclareMathOperator*{\argmin}{arg\,min} % thin space, limits underneath in displays
\DeclareMathOperator{\argmin}{argmin} % no space, limits underneath in displays



% Packages hyperref and algorithmic misbehave sometimes.  We can fix
% this with the following command.

\newcommand{\Expect}{\mathds{E}} %{{\rm I\kern-.3em E}}
\newcommand{\probability}{\mathds{P}} %{{\rm I\kern-.3em P}}

\newtheorem{proposition}{Proposition}

\newcommand\modt{\stackrel{\mathclap{\normalfont 2}}{\equiv}}


\title{Supplement to: Inferring Graphics Programs from Images}

% The \author macro works with any number of authors. There are two
% commands used to separate the names and addresses of multiple
% authors: \And and \AND.
%
% Using \And between authors leaves it to LaTeX to determine where to
% break the lines. Using \AND forces a line break at that point. So,
% if LaTeX puts 3 of 4 authors names on the first line, and the last
% on the second line, try using \AND instead of \And before the third
% author name.

\author{
Kevin Ellis \\
Brain and Cognitive Sciences\\
MIT\\
%Pittsburgh, PA 15213 \\
\texttt{ellisk@mit.edu} \\
Daniel Ritchie\\
Department of Computer Science\\
Brown
\And
Armando Solar-Lezama \\
  CSAIL\\
MIT \\
\texttt{asolar@csail.mit.edu} \\
\And
Joshua B. Tenenbaum \\
Brain and Cognitive Sciences\\
MIT\\
\texttt{jbt@mit.edu} \\
}

\begin{document}
% \nipsfinalcopy is no longer used

\maketitle

\section{Neural network architecture}

\subsection{Convolutional network}
The convolutional network takes as input 2 $256\times 256$ images
represented as a $2\times 256\times 256\times$ volume. These are
passed through two layers of convolutions separated by ReLU
nonlinearities and max pooling:
\begin{itemize}
\item Layer 1: 20 $8\times 8$ convolutions, 2 $16\times 4$ convolutions, 2 $4\times 16$ convolutions. Followed by $8\times 8$ pooling with a stride size of 4.
\item Layer 2: 10 $8\times 8$ convolutions. Followed by $4\times 4$ pooling with a stride size of 4.
\end{itemize}
Training takes a little bit less than a day on a Nvidia TitanX GPU.
The network was trained on $10^5$ synthetic examples.

\subsection{Autoregressive decoding of drawing commands}

Given the image features $f$, we predict the first token using logistic regression:
\begin{equation}
  \probability [T_1]\propto W_{T_1}f
\end{equation}
where $W_{T_1}$ is a learned weight matrix.

Subsequent tokens are predicted as:
\begin{equation}
  \probability [T_n|T_{1:(n - 1)}]\propto \text{MLP}_{T_1,n}(I \otimes \bigotimes_{j < n} \text{oneHot}(T_j))
\end{equation}
Thus each token of each drawing primitive has its own learned MLP.
For predicting the coordinates of lines we found that using 32 hidden nodes with sigmoid activations worked well;
for other tokens the MLP's are just logistic regression (no hidden nodes).

\subsection{A learned likelihood surrogate}

Our architecture for
$L_{\text{learned}}(\text{render}(T_1)|\text{render}(T_2))$ has the
same series of convolutions as the network that predicts the next
drawing command. We train it to predict two scalars: $|T_1 - T_2|$ and
$|T_2 - T_1|$.  These predictions are made using linear regression
from the image features followed by a ReLU nonlinearity; this
nonlinearity makes sense because the predictions can never be negative
but could be arbitrarily large positive numbers.

We train this network by sampling random synthetic scenes for $T_1$,
and then perturbing them in small ways to produce $T_2$.
We minimize the squared loss between the network's prediction and the ground truth symmetric differences.
$T_1$ is rendered in a ``simulated hand drawing'' style which we describe next.

\section{Simulating hand drawings}

We introduce noise into the rendering process by:

\begin{itemize}
\item Rescaling the image intensity by a factor chosen uniformly at random from $[0.5,1.5]$
\item Translating the image by $\pm 3$ pixels chosen uniformly random
\item Rendering the \LaTeX~using the \verb|pencildraw| style,
  which adds random perturbations to the paths drawn by \LaTeX in a way designed to resemble a pencil.
\item Randomly perturbing the positions and sizes of primitive  \LaTeX drawing commands
  \end{itemize}

%% \section{Neural networks for guiding SMC}



%% Let $L(\cdot | \cdot):\text{image}^2\to \mathcal{R}$ be our likelihood
%% function: it takes two images, an observed target image and a
%% hypothesized program output, and gives the likelihood of the observed
%% image conditioned on the program output. We want to sample from:
%% \begin{equation}
%% \probability [p|x]  \propto L(x | \text{render}(p)) \probability [p]
%% \end{equation}
%% where $\probability [p]$ is the prior probability of program $p$, and $x$ is the observed image.

%% Let $p$ be a program with $L$ lines, which we will write as $p = (p_1,p_2,\cdots,p_L)$. Assume the prior factors into:
%% \begin{equation}
%%   \probability [p]\propto \prod_{l\leq L}\probability [p_l]
%% \end{equation}
%% Define the distribution $q_L(\cdot)$, which happens to be proportional to the above posterior:
%% \begin{equation}
%%   q_L(p_1,p_2,\cdots,p_{L - 1},p_L)\propto q_{L - 1}(p_1,p_2,\cdots,p_{L - 1})\times \frac{L(x | \text{render}(p_1,p_2,\cdots,p_{L - 1},p_L))}{L(x | \text{render}(p_1,p_2,\cdots,p_{L - 1}))}\times\probability [p_L]
%% \end{equation}
%% Now suppose we have some samples from $q_{L - 1}(\cdot)$, and that we
%% then sample a $p_L$ from a distribution proportional to $\frac{L(x |
%%   \text{render}(p_1,p_2,\cdots,p_{L - 1},p_L))}{L(x |
%%   \text{render}(p_1,p_2,\cdots,p_{L - 1}))}\times\probability [p_L]$.
%% The resulting programs $p$ are distributed according to $q_L$, and so
%% are also distributed according to $\probability [p|x]$.

%% How do we sample $p_L$ from a distribution proportional to $\frac{L(x
%%   | \text{render}(p_1,p_2,\cdots,p_{L - 1},p_L))}{L(x |
%%   \text{render}(p_1,p_2,\cdots,p_{L - 1}))}\times\probability [p_L]$?
%% We have a neural network that takes as input the target image $x$ and
%% the program so far, and produces a distribution over next lines of
%% code ($p_L$).  We write $\text{NN}(p_L | p_1,\cdots,p_{L - 1};x)$ for
%% the distribution output by the neural network. So we can sample from NN and then weight the samples by:
%% \begin{equation}
%%   w(p_L) = \frac{\probability [p_L]}{\text{NN}(p_L | p_1,\cdots,p_{L - 1};x)}\times \frac{L(x | \text{render}(p_1,p_2,\cdots,p_{L - 1},p_L))}{L(x | \text{render}(p_1,p_2,\cdots,p_{L - 1}))}
%% \end{equation}
%% Then we can resample from these now weighted samples to get a new
%% population of particles (here programs are particles), where each
%% program now has $L$ lines instead of $L - 1$.

%% This procedure can be seen as a particle filter, where each successive
%% latent variable is another line of code, and the emission
%% probabilities are successive ratios of likelihoods under $L(\cdot |
%% \cdot)$.


%%   \begin{algorithm}[tb]
%%    \caption{Neurally guided SMC}
%%    \label{guideAlgorithm}
%% \begin{algorithmic}
%%   \STATE {\bfseries Input:} Neural network NN, beam size $N$, maximum length $L$, target image $x$
%%   \STATE {\bfseries Output:} Samples of the program trace
%%   \STATE Set $B_0 = \{\text{empty program}\}$
%%   \FOR{$1\leq l\leq L$}
%%   \FOR{$1\leq n\leq N$}
%%   \STATE{ $p_n\sim \text{Uniform}(B_{l - 1})$}
%%   \STATE{ $p'_{n}\sim \text{NN}(\text{render}(p),x)$}
%%   \STATE{ Define $r_n = p'_n\cdot p_n$}
%%   \STATE{ Set $\tilde{w}(r_n) = \frac{L(x|r_n)}{L(x|p_n)}\times\frac{\probability [p'_n]}{\probability [p'_n = \text{NN}(\text{render}(p),x)]}$}
%%   \ENDFOR
%%   \STATE{ Define $w(p) = \frac{\tilde{w}(p)}{\sum_{p'}\tilde{w}(p')}$}
%%   \STATE{ Set $B_l$ to be $N$ samples from $r_n$ distributed according to $w(\cdot)$}
%%   \ENDFOR
%%   \STATE {\bfseries return} $\{p : p\in B_{l\leq L}, p \text{ is finished}\}$
%% \end{algorithmic}
%%   \end{algorithm}

\section{Full results on drawings data set}


        \begin{tabular}{ll}
\includegraphics[width = 5cm]{../TikZ/drawings/expert-41.png}&
        \begin{minipage}{10cm}
        \begin{verbatim}
  Rectangle(0,0,1,6)
  Rectangle(2,0,3,6)
  Rectangle(4,0,5,6)
        \end{verbatim}
\end{minipage}
\end{tabular}        
        \\

        \begin{tabular}{ll}
\includegraphics[width = 5cm]{../TikZ/drawings/expert-17.png}&
        \begin{minipage}{10cm}
        \begin{verbatim}
  Rectangle(2,9,4,11)
    for (3)
        Line(-2*i + 7,3*i + 3,-2*i + 6,3*i + 1,arrow = True,solid = True)
        Line(2*i + 3,-3*i + 9,2*i + 4,-3*i + 7,arrow = True,solid = True)
        Rectangle(2*i,-3*i + 6,2*i + 2,-3*i + 8)
        Rectangle(-2*i + 8,3*i,-2*i + 10,3*i + 2)
        \end{verbatim}
\end{minipage}
\end{tabular}        
        \\

        \begin{tabular}{ll}
\includegraphics[width = 5cm]{../TikZ/drawings/expert-43.png}&
        \begin{minipage}{10cm}
        \begin{verbatim}
      Line(0,0,0,5,arrow = False,solid = True)
        \end{verbatim}
\end{minipage}
\end{tabular}        
        \\

        \begin{tabular}{ll}
\includegraphics[width = 5cm]{../TikZ/drawings/expert-63.png}&
        \begin{minipage}{10cm}
        \begin{verbatim}
  Rectangle(0,0,5,5)
  Rectangle(1,1,4,4)
  Rectangle(2,2,3,3)
        \end{verbatim}
\end{minipage}
\end{tabular}        
        \\

        \begin{tabular}{ll}
\includegraphics[width = 5cm]{../TikZ/drawings/expert-11.png}&
        \begin{minipage}{10cm}
        \begin{verbatim}
  Circle(1,1)
        \end{verbatim}
\end{minipage}
\end{tabular}        
        \\

        \begin{tabular}{ll}
\includegraphics[width = 5cm]{../TikZ/drawings/expert-45.png}&
        \begin{minipage}{10cm}
        \begin{verbatim}
  Rectangle(0,4,4,8)
        reflect(y = 12)
        Circle(7,6)
        Line(2,2,2,4,arrow = True,solid = True)
        Line(4,6,6,6,arrow = True,solid = True)
        Rectangle(1,0,3,2)
        \end{verbatim}
\end{minipage}
\end{tabular}        
        \\

        \begin{tabular}{ll}
\includegraphics[width = 5cm]{../TikZ/drawings/expert-51.png}&
        \begin{minipage}{10cm}
        \begin{verbatim}
  Rectangle(2,2,5,3)
  Rectangle(0,0,3,1)
  Rectangle(4,4,7,5)
        \end{verbatim}
\end{minipage}
\end{tabular}        
        \\

        \begin{tabular}{ll}
\includegraphics[width = 5cm]{../TikZ/drawings/expert-21.png}&
        \begin{minipage}{10cm}
        \begin{verbatim}
  Circle(1,5)
    for (2)
        Line(-5*i + 9,-1*i + 2,-7*i + 9,-3*i + 4,arrow = True,solid = True)
        Rectangle(-4*i + 8,4*i,-4*i + 10,4*i + 2)
              reflect(x = 6)
              Line(7*i + 1,-1*i + 2,5*i + 1,-3*i + 4,arrow = True,solid = True)
              Rectangle(8*i,4*i,8*i + 2,4*i + 2)
        \end{verbatim}
\end{minipage}
\end{tabular}        
        \\

        \begin{tabular}{ll}
\includegraphics[width = 5cm]{../TikZ/drawings/expert-2.png}&
        \begin{minipage}{10cm}
        \begin{verbatim}
  Rectangle(4,2,6,5)
        reflect(y = 7)
        Line(2,6,4,4,arrow = True,solid = True)
        Rectangle(0,5,2,7)
        \end{verbatim}
\end{minipage}
\end{tabular}        
        \\

        \begin{tabular}{ll}
\includegraphics[width = 5cm]{../TikZ/drawings/expert-24.png}&
        \begin{minipage}{10cm}
        \begin{verbatim}
  Line(3,3,3,2,arrow = True,solid = True)
  Rectangle(0,7,6,9)
  Rectangle(2,0,4,2)
  Rectangle(0,3,6,6)
        Line(1,7,1,6,arrow = True,solid = True)
        \end{verbatim}
\end{minipage}
\end{tabular}        
        \\

        \begin{tabular}{ll}
\includegraphics[width = 5cm]{../TikZ/drawings/expert-7.png}&
        \begin{minipage}{10cm}
        \begin{verbatim}
  for (3)
      Circle(-3*i + 7,1)
      Circle(3*i + 1,6)
      Line(3*i + 1,2,3*i + 1,5,arrow = False,solid = True)
        \end{verbatim}
\end{minipage}
\end{tabular}        
        \\

        \begin{tabular}{ll}
\includegraphics[width = 5cm]{../TikZ/drawings/expert-44.png}&
        \begin{minipage}{10cm}
        \begin{verbatim}
      reflect(x = 12)
      Circle(4,1)
      Line(9,1,10,1,arrow = False,solid = True)
      Rectangle(10,0,12,2)
        \end{verbatim}
\end{minipage}
\end{tabular}        
        \\

        \begin{tabular}{ll}
\includegraphics[width = 5cm]{../TikZ/drawings/expert-64.png}&
        \begin{minipage}{10cm}
        \begin{verbatim}
      reflect(x = 6)
      Line(5,2,5,4,arrow = False,solid = True)
            reflect(y = 6)
            Line(2,5,4,5,arrow = False,solid = True)
            Rectangle(0,4,2,6)
        \end{verbatim}
\end{minipage}
\end{tabular}        
        \\

        \begin{tabular}{ll}
\includegraphics[width = 5cm]{../TikZ/drawings/expert-36.png}&
        \begin{minipage}{10cm}
        \begin{verbatim}
  for (2)
      Rectangle(0,-3*i + 8,1,-2*i + 9)
      Rectangle(-3*i + 8,5*i,9,8*i + 1)
      Rectangle(-7*i + 7,-2*i + 2,-5*i + 9,4)
        \end{verbatim}
\end{minipage}
\end{tabular}        
        \\

        \begin{tabular}{ll}
\includegraphics[width = 5cm]{../TikZ/drawings/expert-58.png}&
        \begin{minipage}{10cm}
        \begin{verbatim}
  Rectangle(4,0,5,4)
    for (2)
        Line(8,0,-8*i + 8,-7*i + 7,arrow = True,solid = True)
        Rectangle(-4*i + 6,0,-4*i + 7,-2*i + 5)
        \end{verbatim}
\end{minipage}
\end{tabular}        
        \\

        \begin{tabular}{ll}
\includegraphics[width = 5cm]{../TikZ/drawings/expert-3.png}&
        \begin{minipage}{10cm}
        \begin{verbatim}
  Circle(10,5)
  Line(7,5,9,5,arrow = True,solid = True)
  Rectangle(5,3,7,7)
  Rectangle(0,0,12,10)
        reflect(y = 10)
        Line(3,2,5,4,arrow = True,solid = True)
        Rectangle(1,1,3,3)
        \end{verbatim}
\end{minipage}
\end{tabular}        
        \\

        \begin{tabular}{ll}
\includegraphics[width = 5cm]{../TikZ/drawings/expert-28.png}&
        \begin{minipage}{10cm}
        \begin{verbatim}
  Line(0,0,0,2,arrow = False,solid = True)
  Line(0,2,2,2,arrow = False,solid = True)
        \end{verbatim}
\end{minipage}
\end{tabular}        
        \\

        \begin{tabular}{ll}
\includegraphics[width = 5cm]{../TikZ/drawings/expert-22.png}&
        \begin{minipage}{10cm}
        \begin{verbatim}
  Line(8,5,6,5,arrow = True,solid = True)
  Line(4,5,2,5,arrow = True,solid = True)
    for (3)
        Line(-4*i + 9,4,-4*i + 9,2,arrow = True,solid = True)
        Rectangle(4*i,0,4*i + 2,2)
        Rectangle(-4*i + 8,4,-4*i + 10,6)
        \end{verbatim}
\end{minipage}
\end{tabular}        
        \\

        \begin{tabular}{ll}
\includegraphics[width = 5cm]{../TikZ/drawings/expert-69.png}&
        \begin{minipage}{10cm}
        \begin{verbatim}
  Rectangle(0,4,5,6)
        reflect(x = 5)
        Circle(4,1)
        Line(1,4,1,2,arrow = True,solid = True)
        \end{verbatim}
\end{minipage}
\end{tabular}        
        \\

        \begin{tabular}{ll}
\includegraphics[width = 5cm]{../TikZ/drawings/expert-14.png}&
        \begin{minipage}{10cm}
        \begin{verbatim}
  Rectangle(0,6,1,7)
    for (3)
        Rectangle(-2*i + 6,2*i,-2*i + 7,2*i + 1)
        Rectangle(-2*i + 4,2*i,-2*i + 5,2*i + 1)
        \end{verbatim}
\end{minipage}
\end{tabular}        
        \\

        \begin{tabular}{ll}
\includegraphics[width = 5cm]{../TikZ/drawings/expert-68.png}&
        \begin{minipage}{10cm}
        \begin{verbatim}
  for (3)
      Circle(4*i + 1,1)
      Rectangle(4*i,0,4*i + 2,2)
        \end{verbatim}
\end{minipage}
\end{tabular}        
        \\

        \begin{tabular}{ll}
\includegraphics[width = 5cm]{../TikZ/drawings/expert-65.png}&
        \begin{minipage}{10cm}
        \begin{verbatim}
      reflect(y = 6)
      Line(2,1,4,1,arrow = False,solid = True)
            reflect(x = 6)
            Circle(1,1)
            Line(1,2,1,4,arrow = False,solid = True)
        \end{verbatim}
\end{minipage}
\end{tabular}        
        \\

        \begin{tabular}{ll}
\includegraphics[width = 5cm]{../TikZ/drawings/expert-67.png}&
        \begin{minipage}{10cm}
        \begin{verbatim}
  Line(1,5,5,1,arrow = False,solid = True)
  Line(1,4,5,0,arrow = False,solid = True)
  Rectangle(5,0,6,1)
  Rectangle(0,4,1,5)
        \end{verbatim}
\end{minipage}
\end{tabular}        
        \\

        \begin{tabular}{ll}
\includegraphics[width = 5cm]{../TikZ/drawings/expert-12.png}&
        \begin{minipage}{10cm}
        \begin{verbatim}
  for (2)
      Circle(-5*i + 6,1)
      Line(-4*i + 6,-1*i + 2,-1*i + 6,-4*i + 5,arrow = False,solid = True)
      Line(-1*i + 2,-4*i + 6,-4*i + 5,-1*i + 6,arrow = False,solid = True)
      Rectangle(5*i,5,5*i + 2,7)
        \end{verbatim}
\end{minipage}
\end{tabular}        
        \\

        \begin{tabular}{ll}
\includegraphics[width = 5cm]{../TikZ/drawings/expert-42.png}&
        \begin{minipage}{10cm}
        \begin{verbatim}
  Line(0,0,0,5,arrow = False,solid = False)
  Line(4,1,4,5,arrow = False,solid = False)
  Line(4,0,4,1,arrow = False,solid = False)
        \end{verbatim}
\end{minipage}
\end{tabular}        
        \\

        \begin{tabular}{ll}
\includegraphics[width = 5cm]{../TikZ/drawings/expert-10.png}&
        \begin{minipage}{10cm}
        \begin{verbatim}
  Rectangle(0,0,3,4)
        \end{verbatim}
\end{minipage}
\end{tabular}        
        \\

        \begin{tabular}{ll}
\includegraphics[width = 5cm]{../TikZ/drawings/expert-35.png}&
        \begin{minipage}{10cm}
        \begin{verbatim}
  Circle(1,1)
    for (2)
        Circle(-5*i + 6,5*i + 1)
        Line(1,5,5*i + 1,2,arrow = True,solid = True)
        \end{verbatim}
\end{minipage}
\end{tabular}        
        \\

        \begin{tabular}{ll}
\includegraphics[width = 5cm]{../TikZ/drawings/expert-31.png}&
        \begin{minipage}{10cm}
        \begin{verbatim}
  Circle(7,3)
    for (3)
        Circle(-3*i + 7,5)
        Circle(-3*i + 7,2*i + 1)
        Rectangle(-3*i + 6,2*i,-3*i + 8,6)
        \end{verbatim}
\end{minipage}
\end{tabular}        
        \\

        \begin{tabular}{ll}
\includegraphics[width = 5cm]{../TikZ/drawings/expert-37.png}&
        \begin{minipage}{10cm}
        \begin{verbatim}
  Circle(1,8)
    for (2)
        Line(4*i + 1,-5*i + 7,2*i + 3,5,arrow = False,solid = True)
        Rectangle(-4*i + 4,5*i,6,7*i + 2)
        \end{verbatim}
\end{minipage}
\end{tabular}        
        \\

        \begin{tabular}{ll}
\includegraphics[width = 5cm]{../TikZ/drawings/expert-40.png}&
        \begin{minipage}{10cm}
        \begin{verbatim}
  Circle(7,1)
  Circle(4,1)
  Circle(1,1)
        \end{verbatim}
\end{minipage}
\end{tabular}        
        \\

        \begin{tabular}{ll}
\includegraphics[width = 5cm]{../TikZ/drawings/expert-29.png}&
        \begin{minipage}{10cm}
        \begin{verbatim}
  Line(1,2,1,5,arrow = False,solid = True)
    for (2)
        Line(2,-4*i + 4,-4*i + 6,4,arrow = False,solid = True)
        Line(0,-2*i + 6,-2*i + 2,6,arrow = False,solid = True)
        Line(1,5,2*i + 2,5,arrow = False,solid = True)
        \end{verbatim}
\end{minipage}
\end{tabular}        
        \\

        \begin{tabular}{ll}
\includegraphics[width = 5cm]{../TikZ/drawings/expert-62.png}&
        \begin{minipage}{10cm}
        \begin{verbatim}
  Rectangle(5,0,8,3)
  Rectangle(2,1,4,3)
  Rectangle(0,2,1,3)
        \end{verbatim}
\end{minipage}
\end{tabular}        
        \\

        \begin{tabular}{ll}
\includegraphics[width = 5cm]{../TikZ/drawings/expert-57.png}&
        \begin{minipage}{10cm}
        \begin{verbatim}
  for (3)
      Circle(-4*i + 9,4)
      Circle(-4*i + 9,1)
      Circle(4*i + 1,7)
        \end{verbatim}
\end{minipage}
\end{tabular}        
        \\

        \begin{tabular}{ll}
\includegraphics[width = 5cm]{../TikZ/drawings/expert-55.png}&
        \begin{minipage}{10cm}
        \begin{verbatim}
  Circle(1,5)
  Line(1,4,1,2,arrow = True,solid = True)
  Rectangle(0,0,2,2)
        \end{verbatim}
\end{minipage}
\end{tabular}        
        \\

        \begin{tabular}{ll}
\includegraphics[width = 5cm]{../TikZ/drawings/expert-66.png}&
        \begin{minipage}{10cm}
        \begin{verbatim}
  Line(0,2,7,2,arrow = False,solid = True)
  Line(1,1,6,1,arrow = False,solid = True)
  Line(2,0,5,0,arrow = False,solid = True)
        \end{verbatim}
\end{minipage}
\end{tabular}        
        \\

        \begin{tabular}{ll}
\includegraphics[width = 5cm]{../TikZ/drawings/expert-26.png}&
        \begin{minipage}{10cm}
        \begin{verbatim}
  Circle(1,9)
  Line(1,8,1,6,arrow = True,solid = True)
  Line(1,3,1,2,arrow = True,solid = True)
  Line(1,3,1,4,arrow = False,solid = True)
        reflect(y = 6)
        Circle(1,1)
        \end{verbatim}
\end{minipage}
\end{tabular}        
        \\

        \begin{tabular}{ll}
\includegraphics[width = 5cm]{../TikZ/drawings/expert-15.png}&
        \begin{minipage}{10cm}
        \begin{verbatim}
  Line(1,2,3,2,arrow = False,solid = True)
  Line(2,1,4,1,arrow = False,solid = False)
  Line(3,0,5,0,arrow = False,solid = True)
  Line(0,3,2,3,arrow = False,solid = False)
        \end{verbatim}
\end{minipage}
\end{tabular}        
        \\

        \begin{tabular}{ll}
\includegraphics[width = 5cm]{../TikZ/drawings/expert-19.png}&
        \begin{minipage}{10cm}
        \begin{verbatim}
  Line(4,4,2,2,arrow = True,solid = True)
  Rectangle(3,4,5,6)
  Rectangle(0,0,2,2)
        \end{verbatim}
\end{minipage}
\end{tabular}        
        \\

        \begin{tabular}{ll}
\includegraphics[width = 5cm]{../TikZ/drawings/expert-8.png}&
        \begin{minipage}{10cm}
        \begin{verbatim}
  Line(0,0,0,4,arrow = False,solid = True)
        \end{verbatim}
\end{minipage}
\end{tabular}        
        \\

        \begin{tabular}{ll}
\includegraphics[width = 5cm]{../TikZ/drawings/expert-23.png}&
        \begin{minipage}{10cm}
        \begin{verbatim}
  Line(4,1,2,1,arrow = False,solid = True)
    for (2)
        Line(-4*i + 5,2,-4*i + 5,4,arrow = True,solid = True)
          for (3)
              Circle(-4*j + 9,4*i + 1)
              Line(8,1,3*i + 6,3*i + 1,arrow = True,solid = False)
        \end{verbatim}
\end{minipage}
\end{tabular}        
        \\

        \begin{tabular}{ll}
\includegraphics[width = 5cm]{../TikZ/drawings/expert-60.png}&
        \begin{minipage}{10cm}
        \begin{verbatim}
  Rectangle(0,3,2,5)
    for (2)
        Circle(3*i + 4,1)
        Circle(-2*i + 7,-3*i + 7)
        Line(5,3,-3*i + 7,2,arrow = True,solid = True)
        Line(4*i + 2,3*i + 4,4,4,arrow = True,solid = True)
        \end{verbatim}
\end{minipage}
\end{tabular}        
        \\

        \begin{tabular}{ll}
\includegraphics[width = 5cm]{../TikZ/drawings/expert-16.png}&
        \begin{minipage}{10cm}
        \begin{verbatim}
  Circle(9,1)
    for (3)
        Circle(2*i + 3,-3*i + 10)
        Circle(-2*i + 5,3*i + 1)
        Line(2*i + 2,-3*i + 7,2*i + 3,-3*i + 9,arrow = False,solid = True)
        Line(-2*i + 7,3*i + 3,-2*i + 8,3*i + 1,arrow = False,solid = True)
        \end{verbatim}
\end{minipage}
\end{tabular}        
        \\

        \begin{tabular}{ll}
\includegraphics[width = 5cm]{../TikZ/drawings/expert-13.png}&
        \begin{minipage}{10cm}
        \begin{verbatim}
  Line(5,0,3,0,arrow = True,solid = True)
  Line(0,3,2,3,arrow = True,solid = True)
  Line(3,2,1,2,arrow = True,solid = True)
  Line(2,1,4,1,arrow = True,solid = True)
        \end{verbatim}
\end{minipage}
\end{tabular}        
        \\

        \begin{tabular}{ll}
\includegraphics[width = 5cm]{../TikZ/drawings/expert-47.png}&
        \begin{minipage}{10cm}
        \begin{verbatim}
      reflect(y = 11)
      Rectangle(4,9,7,10)
            reflect(x = 11)
            Rectangle(8,0,11,3)
            Rectangle(1,4,2,7)
        \end{verbatim}
\end{minipage}
\end{tabular}        
        \\

        \begin{tabular}{ll}
\includegraphics[width = 5cm]{../TikZ/drawings/expert-5.png}&
        \begin{minipage}{10cm}
        \begin{verbatim}
  Line(12,9,12,0,arrow = True,solid = True)
    for (2)
        Rectangle(-3*i + 6,3*i + 5,-3*i + 8,9)
        Rectangle(9*i,-4*i + 7,9*i + 2,9)
        \end{verbatim}
\end{minipage}
\end{tabular}        
        \\

        \begin{tabular}{ll}
\includegraphics[width = 5cm]{../TikZ/drawings/expert-59.png}&
        \begin{minipage}{10cm}
        \begin{verbatim}
  Line(4,0,0,0,arrow = False,solid = False)
        \end{verbatim}
\end{minipage}
\end{tabular}        
        \\

        \begin{tabular}{ll}
\includegraphics[width = 5cm]{../TikZ/drawings/expert-50.png}&
        \begin{minipage}{10cm}
        \begin{verbatim}
  Line(8,3,9,3,arrow = False,solid = True)
    for (2)
        Line(-6*i + 8,3*i + 3,-3*i + 7,3,arrow = True,solid = True)
        Line(5*i + 2,3*i + 1,5*i + 4,3,arrow = True,solid = True)
        Rectangle(9*i,2*i,10*i + 2,3*i + 2)
        Rectangle(-4*i + 4,3*i + 2,-5*i + 7,2*i + 5)
        \end{verbatim}
\end{minipage}
\end{tabular}        
        \\

        \begin{tabular}{ll}
\includegraphics[width = 5cm]{../TikZ/drawings/expert-6.png}&
        \begin{minipage}{10cm}
        \begin{verbatim}
  Line(0,1,0,4,arrow = False,solid = True)
  Rectangle(2,0,5,3)
    for (2)
        Line(0,3*i + 1,2,3*i,arrow = False,solid = True)
        Line(-3*i + 3,4,-2*i + 5,3,arrow = False,solid = True)
        \end{verbatim}
\end{minipage}
\end{tabular}        
        
t
\end{document}
